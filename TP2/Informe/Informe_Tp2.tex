\documentclass[11pt, a4paper]{article}

% Configuración de márgenes de las páginas
	%\usepackage[left=3cm,right=3cm,top=2cm, bottom=2cm]{geometry}
	\usepackage{a4wide}

% Paquete de acentos para Linux
	\usepackage[utf8]{inputenc}

% Paquete de acentos para windows
%	\usepackage[latin1]{inputenc}

% Paquete para reconocer la separación en sílabas en español
	\usepackage[spanish]{babel}

% Paquetes especiales para el TP
	\usepackage{./otros/caratula}
	\usepackage{./otros/algo2symb}
	\usepackage[lined]{./otros/algorithm2e}
	\usepackage{amssymb}			%simbolos matematicos
	\usepackage{pdfpages}

% Paquete para incluir hypervinculos
	\usepackage{color}
	\usepackage{url}
	\definecolor{lnk}{rgb}{0,0,0.4}
	\usepackage[colorlinks=true,linkcolor=lnk,citecolor=blue,urlcolor=blue]{hyperref}

% Paquete para armar índices
	\usepackage{makeidx}
	\makeindex

% Más espacio entre líneas
	\parskip=1.5pt

% Comandos personalizados
	\newcommand{\nat}{\ensuremath{\mathbb{N}}}
	\newcommand{\entero}{\ensuremath{\mathbb{Z}}}
	\newcommand{\real}{\ensuremath{\mathbb{R}}}
	\newcommand{\grafo}{\ensuremath{\mathbb{grafo}}}
	\newcommand{\titade}[1]{\ensuremath{\Theta(#1)}}
	\newcommand{\omegade}[1]{\ensuremath{\Omega(#1)}}


\begin{document}

% Carátula
	\titulo{Primer Trabajo Práctico}
	\fecha{14 de Abril de 2010}
	\materia{Algoritmos y Estructuras de Datos III}
	\integrante{Bianchi, Mariano}{92/08}{bianchi-mariano@hotmail.com}
	\integrante{Brusco, Pablo}{527/08}{pablo.brusco@gmail.com}
	\integrante{Di Pietro, Carlos Augusto Lyon}{126/08}{cdipietro@dc.uba.ar}
	\maketitle

% Índice
\newpage \section{Índice} \printindex \tableofcontents

% Cuerpo del informe
\newpage	 	\section*{Ejercicio 1}

\subsection*{Introducción}
	El primer problema del presente trabajo consistió en la implementación de un algoritmo capaz de dar solución a la ecuación 
	\begin{equation} 
		b^n\ mod\ (n) 
	\end{equation} 
haciendo uso de alguna de las técnicas algorítmicas aprendidas hasta el momento en la materia. Asímismo, la consigna dictaba que la complejidad final del algoritmo debería ser menor a \Ode{n}.\\
	En pos de cumplimentar lo pedido se decidió usar la técnica de \textit{Dividir \& Conquistar}\footnote{Poner alguna referencia 	en donde se explique esta técnica} para desarrollar el algoritmo. Esta técnica se caracteriza principalmente en dividir la instancia de un problema en instancias más pequeñas, atacar cada una de ellas por separado y resolverlas, para finalmente juntar sus resultados y así producir el resultado final.

\subsection*{Detalles de implementación}
	La primera solución que se piensa casi de manera intuitiva es la de mutliplicar \textit{n} veces el número \textit{b} y luego hallar el resto de dividir ese resultado por \textit{n}. 
	\begin{equation}
		\underbrace{b.b.b \hdots b.b.b}_{n\ veces}\ mod\ (n)  = b^n\ mod\ (n)
	\end{equation}
Sin embargo, la complejidad ese algoritmo es \Ode{n}, ya que se realizan \textit{n} multipicaciones y 1 división, razón por lo cual no cumple con lo pedido en la consigna.\\
No obstante, la idea anterior conduce a otra forma de encarar el problema. Veamos de qué se trata. \\

	Sabemos que $b^n\ mod\ (n)$ es el resto de dividir $b^n$ por $n$. Llamemos \textit{x} a ese resto. Luego: 
	\begin{equation}
		 x = b^n\ mod\ (n)\ , \hspace{10pt}con\ 0 \leq x < n
	\end{equation}
ya que el Algoritmo de División de Números Enteros asegura que el resto existe, es único y es un valor entre 0 y \textit{n}.\\

	Inmediatamente de lo anterior, por la definición de congruencia, se desprende que:
	\begin{equation}
		b^n\ \equiv\ x\ (n)
	\label{primera_idea}
	\end{equation}

	Luego, una de las propiedades de congruencias, nos asegura que:
	\begin{equation}
		a \equiv p (n)\ ,\ b \equiv q (n)\ \Longrightarrow\ a.b \equiv p.q (n)\ , \hspace{10pt} \forall\ a,b,p,q,n \in \mathbf Z
	\end{equation}
por lo que, tomando en particular, $a=b$ sigue que:
	\begin{equation}
		b \equiv\ p (n)\ \Longrightarrow\ b.b \equiv p.p\  \Leftrightarrow\ b^2 = p^2 (n)
	\label{resto_b2}
	\end{equation}

	Sea ahora \textit{k} el resto de la division de $ p^2 $ por \textit{n}; luego:
	\begin{equation}
		p^n\ \equiv\ k\ (n)\ , \hspace{10pt} con\ 0 \leq k < n
	\label{resto_p2}
	\end{equation}

	Por lo tanto, de \ref{resto_b2} y \ref{resto_p2} se obtiene usando la propiedad transitiva de congruencias que:
	\begin{equation}
		b^2 \equiv p^2 (n)\ ,\ p^2 \equiv k (n)\ \Longrightarrow\ b^2 \equiv k (n)
	\label{segunda_idea}
	\end{equation}
donde por \ref{resto_p2} sabíamos que $k$ es un número entre 0 y $n$.\\

	De este modo, podemos aplicar el resultado de \ref{segunda_idea} en \ref{primera_idea}, obteniéndose:
	\begin{equation}
		b^n = \underbrace{b.b.b \hdots b.b.b}_{n\ veces}\ \equiv\ x (n) \\
		b^n = \underbrace{b^2. b^2 \hdots b^2. b^2}_{n/2\ veces}\ \equiv\ x (n) \\
		b^n = \underbrace{k. k \hdots k. k}_{n/2\ veces}\ \equiv\ x (n)
	\label{idea_final}
	\end{equation}
	

%\subsection*{Análisis de la complejidad del algoritmo}
%\subsection*{Debate}
%\subsection*{Comentarios}
%\subsection*{Conclusiones}

\newpage 	\section{Ejercicio 2}

\subsection{Introducción}
\label{intro2}

\paragraph{}
El segundo problema que se propone en este trabajo práctico es el de, dado el trazado de calles de una ciudad compueto por esquinas y calles, decidir si es posible orientar las calles de forma tal que dadas dos esquinas cualesquiera resulte posible llegar de la primera a la segunda.

\paragraph{}
Mediante un modelado con grafos, el problema puede ser representado de forma que los nodos de un grafo correspondan a las esquinas de la ciudad y las aristas a las calles que unen cada par de esquinas entre sí. \\
Entonces, para poder estudiar el problema con el modelo planteado es preciso primero establecer algunas definiciones:
\begin{itemize}
	\item \textbf{Def$_1$:} Un digrafo es ...
	\item \textbf{Def$_2$:} Un digrafo es \textit{fuertemente conexo} ...
	\item \textbf{Def$_3$:} Un grafo $G$ es \textit{fuertemente orientable} si existe una asignación de direcciones a los ejes del conjunto de ejes del grafo $G$ tal que el digrafo resultante es fuertemente conexo. 
	\item \textbf{Def$_4$:} \textit{un puente o arista de corte} es una arista que al eliminarse de un grafo incrementa el número de componentes conexos de éste. Equivalentemente, una arista es un puente si y sólo si no está contenido en ningún ciclo.
\end{itemize}

\paragraph{}
A partir de las definiciones anteriores, se desprende que para resolver el problema en cuestión se debe determinar si el grafo que modela el trazado de las calles es fuertemente orientable. Esto es, establecer si es posible para cada instancia del problema orientar el grafo dado para obtener así un digrafo fuertemente conexo de modo que sea posible llegar de cualquier nodo a cualquier otro.

\paragraph{}
El algoritmo desarrollado para dar solución al problema planteado recorre los ejes del grafo evaluando cierta propiedad que se explicará más adelante. Para ello, el mimos, se basa los algoritmos tradicionales de recorrido de grafos, como son BFS y DFS. En la sección contigua se expone de forma detallada las ideas que dieron lugar al algoritmo así como también el algoritmo en sí.

	
\subsection{Explicación}
\label{exp2}

\paragraph{}
Tal como se expuso en la introducción de este informe, el algoritmo a implementar debía ser capaz de decidir si el grafo que modela el trazado de la ciudad es orientable de modo que dos nodos cualesquiera estén conectados.

\paragraph{}
De un primer análisis se desprende que el hecho de que cualquier par de nodos esté conectado implica necesariamente que el grafo sea conexo, ya que de no serlo existiría al menos un nodo no accesible desde alguno de los nodos restantes.

\begin{center}
grafo explicativo
\end{center}

Por otra parte, de existir una orientación del grafo de manera que cualquier par de nodos \textit{a}, \textit{b} esté conectado, eso a su vez implica que existen por lo menos dos caminos que unen a ambos. Uno dirigido de a a b y otro dirigido de b a a.
se dijo también en la introducción al pretender encontrar un digrafo a partir del  grafo original, es preciso que se pueda establecer un 

\underline{Teorema (Robbins, 1939) :}\\
Un grafo conexo G es fuertemente orientable si y solo si G no tiene puentes \footnote{Véase la demotración del teorema en la sección Anexos}.
\label{robbins}

\paragraph{} 
Con esté teorema podemos ver que si encontramos al menos 1 puente en nuestro grafo, significa que no podremos orientarlo como queremos, y de lo contrario, si encontramos que no hay ningun puente, podremos orientarlo. Por lo tanto, el algoritmo utilizado realiza exactamente esa comprobación. Veamos como trabaja:
 
\vspace*{3cm}

%segundo algoritmo:
\incmargin{1em}
\linesnumbered
\restylealgo{boxed}

\textbf{comprobación(Grafo G)}\\
\SetKw{Orden}{Complejidad:}
	\begin{algorithm}[H]
	\Orden{O($n^3$)}

    \textbf{var} eje : int $\leftarrow$ 0 \\
    \textbf{var} n : int $\leftarrow$ cantNodos(G) \\
				


     \While{eje $\leq$ m}{
      k $\leftarrow$ RecorridoSinEje(eje,G)\\

      \lIf{k $\neq$ n}{\textbf{return} no se puede} \\ eje $\leftarrow$ eje $+$ 1} \textbf{return} fuertemente conexo

  \end{algorithm}

RecorridoSinEje(eje,G) es una funcion que recorre el grafo $G$  (con BFS o DFS) sin utilizar la arista $eje$ y retorna la cantidad de nodos visitados, $k$. Como la forma de recorrer utilizada, solo recorre nodos conectados a la raiz (es decir, al nodo donde comienza el recorrido), quiere decir que el resultado k va a ser n (la cantidad total de nodos de $G$) si $G$ es conexo. De lo contrario, si k es menor que $n$, estamos en presencia de 2 componentes conexas (o más en el caso de $eje = 0$). Por lo tanto, el eje sacado, era un puente. 

\paragraph{}
\underline{Aclaración: }
\\
Cuando eje = 0, representa, recorrer a G completo con todos sus nodos. En este punto podria pasar que G no sea conexo y esta función devolvería el resultado correspondiente.


\subsection{Análisis de la complejidad del algoritmo}
\label{complejidad2}

\subsection{Detalles de implementación}
\label{imp2}

\subsection{Resultados}
\label{res2}

\subsection{Debate}
\label{deb2}

\subsection{Conclusiones}
\label{conc2}

\newpage 	\section{Ejercicio 3}

\subsection{Introducción}

\paragraph{}
Para la resolución de este ejercicio se debía desarrollar e implementar una heurística constructiva  para resolver el problema de encontrar un \mc dado un grafo simple.


\subsection{Explicación}

\paragraph{}
En una primera aproximación al problema, se pensó un algoritmo bastante sencillo. La idea del mismo radicaba en ir tomando los nodos en orden de grados, es decir, comenzando con los de mayor grado hasta llegar a los de menor grado. De esta forma, uno puede pensar que al tomar primero los vértices de mayor grado, hay mas chances de encontrar una clique de mayor tamaño.

\paragraph{}
Esto es claramente una heurística válida que utiliza la técnica de algoritmo goloso. Pero es claro también que se pueden encontrar fácilmente ejemplos de grafos en los que dicho algoritmo funcione tan mal como uno quiera.

\paragraph{}
A fines de evitar en cierto grado muchos casos para los cuales este algoritmo funciona mal, se planteó uno nuevo que utiliza la misma idea pero que la misma no se realiza sobre todos los nodos del grafo sino que se hace sobre un subconjunto de los mismos. Para formar dicho conjunto, se implementó un algoritmo que revisa todas las combinaciones de 2 vértices distintos (siempre y cuándo haya 2 o más vertices en el grafo) que sean vecinos entre sí y se guarda en un conjunto de vértices aquellos que sean vecinos a ambos vértices y además se guardan los 2 vértices en cuestión. Esto se repite para cada posible combinación de vértices de a 2, guardando siempre el conjunto más grande que se haya encontrado completado de la forma antes mencionada.

\paragraph{}
Una vez encontrado este subgrafo, se procede a realizar el algoritmo goloso antes mencionado pero sobre dicho subgrafo, es decir, se busca el nodo con mayor grado en ese subgrafo y se coloca en un conjunto, el cuál será devuelto como clique al terminar el algoritmo. Luego, para el resto de los nodos del subgrafo, se va tomando de a uno a la vez en orden de mayor a menor grado (considerando sólo los adyacentes que pertenecen al subgrafo) y se verifica que sea adyacente a todos los que pertenecen a la clique. Si lo es, entonces se lo inserta en el conjunto sino se lo descarta. Finalmente, se prosigue con estos pasos hasta haber intentado con todos los nodos del subgrafo devolviendo entonces la clique encontrada.

\paragraph{}
A continuación se adjunta el pseudocódigo del algoritmo constructivo antes descripto y el de las funciones auxiliares pertinentes.

\vspace{2em}
\incmargin{3em}
\linesnumbered
\restylealgo{boxed}
\footnotesize 
\textbf{AlgoritmoConstructivo}(G: grafo) \\
\begin{algorithm}[H]
	\BlankLine
	frontera = mayorFronteraEnComun(G)\\
	res = $\emptyset$
	\BlankLine

	\textbf{mientras} frontera $\neq\ \emptyset$\\

		\tab v = elMasRelacionado(G,frontera)\\
		\tab \textbf{si} esVecinoDeTodos(G,res,v)\\
			\tab \tab \textbf{insertar} v \textbf{en} res\\
		\tab \textbf{fin si}\\
		\tab \textbf{eliminar} v \textbf{de} frontera\\
	\textbf{fin mientras}\\

	\textbf{si} res $==\ \emptyset$\\
		\tab \textbf{insertar} $v_0$ \textbf{en} res\\
	\textbf{fin si}\\

	\textbf{devolver} res
\caption{Pseudocódigo del algoritmo constructivo}
\end{algorithm}

\vspace{3em}

\textbf{mayorFronteraEnComun}(G: grafo) \\
\begin{algorithm}[H]
	\BlankLine
	aux = $\emptyset$\\
	res = $\emptyset$\\
	\textbf{paratodo} u,v $\in\ V_{G}$ \textbf{tq} (u,v) $\in\ X_{G}$ \\
		\tab aux = (adyacentes(G,u) $\cap$ adyacentes(G,v)) $\cup$ u $\cup$ v\\
		\tab \textbf{si} \#aux $>$ \#res\\
			\tab \tab res = aux\\
		\tab \textbf{fin si}\\
	\textbf{fin paratodo}\\

	\textbf{devolver} res

\caption{Pseudocódigo de un algoritmo secundario al constructivo}
\end{algorithm}
 
\normalsize

\paragraph{}
Comparando el algoritmo goloso por sí sólo y la combinación entre la preselección y el mismo, se puede decir que se mejora en cierto grado el comportamiento del algoritmo ya que existen muchos casos en el que sólo aplicar el algoritmo goloso no daba buenos resultados y que si se aplica la preselección la heurística devuelve mejores resultados, incluso a veces da resultados exactos. Igualmente, al ser una heurística, también existen casos en los que la misma puede funcionar tan mal como uno quiera.


\subsection{Análisis de la complejidad del algoritmo}
\subsection{Detalles de implementación}
\subsection{Resultados}
\subsection{Debate}
\subsection{Conclusiones}

\newpage		\section{Anexos}
\label{anexo}

Para los casos de prueba, se utlizaron grafos generados aleatoriamente de la siguiente manera:
\begin{itemize}
 \item se ingresa cantidad de nodos.
 \item se ingresa una probabilidad de aparición de ejes.
 \item se genera una matriz de adyacencia en donde cada conexión tendra una probabilidad de aparecer igual a la ingresada.
\end{itemize}

De esta manera, no conocemos las cliques generadas, pero es una manera de generar casos aleatorios simples.


\end{document}
