\section{Ejercicio 2}

\subsection{Introducción}
	El segundo problema consistió en la implementación de un algoritmo capaz de dar solución al problema que se plantea a continuación:

	 Decidir si un grupo de n personas pueden formar o no una ronda que cumpla con las siguientes restricciones:
	\begin{itemize}
	      \item La ronda debe contener a todas las personas.
	      \item Algunas personas son amigas y otras no. 
	      \item Cada alumna debe tomar de la mano a dos de sus amigas.
	 
	\end{itemize}

	
\subsection{Explicación}
	Dado que para este problema no se conocen algoritmos buenos\footnote{Un algoritmo se considera bueno si puede ser resuelto en tiempo polinomial.},
	se pensó en utilizar la solución por fuerza bruta, con algunas mejoras, es decir, intentar todas las combinaciones hasta lograr determinar si hay o no solución.
	
	A este método, se le agregó

%\subsection{Análisis de la complejidad del algoritmo}
%\subsection[Detalles de Implementación}
%\subsection{Resultados}
%\subsection{Debate}
%\subsection{Comentarios}
%\subsection{Conclusiones}
