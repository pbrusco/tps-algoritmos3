\section*{Ejercicio 1}

\subsection*{Introducción}
El primer problema del presente trabajo consistió en la implementación de un algoritmo capaz de dar solución a la ecuación \begin{equation} b^n mod (n) \end{equation} haciendo uso de alguna de las técnicas algorítmicas aprendidas hasta el momento en la materia. Asímismo, la consigna dictaba que la complejidad final del algoritmo debería ser menor a \Ode{n}.\\
En pos de cumplimentar lo pedido se decidió usar la técnica de Dividir \& Conquistar\footnote{Poner alguna referencia en donde se explique esta técnica} para desarrollar el algoritmo. Esta técnica se caracteriza principalmente en dividir la instancia de un problema en instancias más pequeñas, atacar cada una de ellas por separado y resolverlas, para finalmente juntar sus resultados y aséi producir el resultado final.

\subsection*{Detalles de implementación}


%\subsection*{Análisis de la complejidad del algoritmo}
%\subsection*{Debate}
%\subsection*{Comentarios}
%\subsection*{Conclusiones}
