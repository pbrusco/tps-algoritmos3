\section{Detalles de implementación}
\paragraph{}
Dentro de la carpeta que contiene todos los archivos de este trabajo práctico, puede encontrar un script de python llamado \textit{tp.py} que tiene por fin facilitar el uso del todos los programa aquí desarrollados. \\
Este script puede ser ejecutado desde una consola, siendo necesario para ello situarse en el directorio en que se encuentra el script e introducir luego la instrucción \texttt{python tp.py}. Una vez ejecutado el script, se presenta en pantalla un menú con las siguientes opciones:
\begin{itemize}
	\item[1] Recompilar el programa todos los programas \footnote{Todos los programas son compilados para sistemas operativos GNU-Linux}.
	\item[2] Recompilar un programa en particular
	\item[3] Ejecutar todos los programas
	\item[4] Ejecutar un programa en particular
	\item[5] Salir
\end{itemize}

\paragraph{}
Una vez compilado/s el/los programa/s que se desea/n ejecutar, eligiendo una de las opciones 2 o 3 respectivamente, se mostrará en pantalla un segundo menú que lista todos los archivos \textit{".in"} ubicados en la carpeta \textit{./in}, permitiendo elegir el aquel que se desee ejecutar o en su defecto, todos ellos. \\
A medida que los archivos \textit{".in"} vayan siendo procesados se mostrará en pantalla el tiempo promedio que tardó cada algoritmo en resolver todas las instancias contenidas en el archivo. Asimismo, una vez finalizada la ejecución de el/los programas
se mostrará un mensaje en pantalla avisando la finalización de el/los mismo/s y se deberá presionar enter para volver al menu inicial. \\
Las respuestas a las instancias de cada archivo \textit{".in"} estarán en un archivo \textit{".out"} homónimo dentro del directorio \textit{./out/Nro\_de\_Ejercicio}

\paragraph{}
Se recomienda que cualquier archivo \textit{".in"} que se desee resolver mediante alguno de los programas sea colocado en la carpeta \textit{./in} para poder ser elegido desde el menu que muestra el script \footnote{En el directorio \textit{./in} hay un archivo ejecutable llamado \textit{generador} que genera archivos con varias instancias de grafos. El mismo puede ejecutarse desde consola por medio del comando \texttt{./generador}}. \\
De cualquier forma, todos los programa también pueden ser compilados por consola con el comando \texttt{make} y pueden ser ejecutados con el comando \texttt{./ej$\#$nro\_de\_ejericio} recibiendo como parámetro las rutas de los archivos de entrada \textit{".in"} a procesar. Pueden recibir tantos archivos como se desee, pero en caso de no recibir ninguno, se procesará el archivo \textit{./in/Tp3.in}. En todos los casos, los archivos correspondientes archivos \textit{".out"} serán guardados en las correspondientes carpetas \textit{./out/Nro\_de\_Ejercicio}.

\paragraph{}
Además, en la carpeta del cd entregado junto con este trabajo se halla una carpeta con el código fuente del generado de grafos utilizado, y una carpeta correspondiente a cada ejercicio. \\
En el interior de cada una de esta últimas se encuentran los archivos fuentes en lenguaje C++ del programa desarrollado, los cuales están cometados para simplificar la comprensión de cada algortimo. Asimismo hay directorios varios con las variantes del codigo utilizado para realizar las mediciones tanto de cantidad de operaciones como de tiempo.
