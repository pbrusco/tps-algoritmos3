\section{Anexos}
\label{anexo}

Para los casos de prueba, se utlizaron grafos generados aleatoriamente de la siguiente manera:
\begin{itemize}
 \item se ingresa cantidad de nodos.
 \item se ingresa una probabilidad de aparición de ejes.
 \item se genera una matriz de adyacencia en donde cada conexión tendra una probabilidad de aparecer igual a la ingresada.
\end{itemize}

De esta manera, no conocemos las cliques generadas, pero es una manera de generar casos aleatorios simples.
