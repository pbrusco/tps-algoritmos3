\section{Ejercicio 2}

\subsection{Introducción}

\paragraph{}
En esta sección se presentara un algoritmo exacto para resolver el problema de encontrar la Clique Máxima en un grafo.


\paragraph{}
Aun no se conocen algoritmos buenos, es decir, polinomiales con respecto al tamaño de la entrada, para resolver 
este problema, asi que nos consentraremos en realizar mejoras al algoritmo de fuerza bruta que considera todos los casos.


\subsection{Explicación}

\paragraph{}
Un algoritmo de fuerza bruta para resolver el problema de Max-Clique podría simplemente intentar formar el conjunto más 
grande de nodos, donde ese conjunto sea completo, intentando todas las posibilidades eligiendo todos los conjuntos de un cierto tamaño,
luego intentar con un tamaño menor, etc. Probablemente la complejidad de un algoritmo de este estilo sea $n^n$ donde n es la cantidad 
de nodos del grafo.

\paragraph{}
Una mejora que surge casi inmediatamente es utilizar la técnica de BackTracking, cuya función principal es intentar podar el 
arbol implicito de combinaciones posibles. 

\paragraph{}
De todas maneras, implementar solo un BackTracking parece ser poco con respecto a las mejoras que se pueden lograr. A continuación, 
se explicara el algoritmo implementado con un pseudocodigo y se verá cada una de las mejoras por separado.

\incmargin{1em}
\linesnumbered
\restylealgo{boxed}

\textbf{Algoritmo Exacto}(G: grafo) \\
\begin{algorithm}[H]
	\BlankLine
		componentesConexas = DetectarComponentesConexas(G)
		Para cada componente en componentesConexas hacer:
			heap = CrearHeapDeGrados(G,componente)
			\BlankLine
			Mientras hayElementos(heap) y top(heap) $\geq$ tamañoCliqueMaxHastaElMomento
			hacer
				v = nodoDeMayorGrado(heap)
				Para 
	
	
			repetir
				
	\caption{Pseudocódigo del algoritmo exacto}
\end{algorithm}



\paragraph{}

